\documentclass[]{article}
\usepackage{url}
\usepackage{marvosym}
\usepackage{hyperref}
\usepackage{amsmath}
\usepackage{amssymb}
\usepackage{float}
	
%opening
\title{}
\author{}

\begin{document}

\maketitle



\section{Priorities}
\subsection{Educational mission}
This activity will  develop infrastructure for distributed processing, that complements the existing support for cloud computing. This infrastructure will enable us to educate the students with hands-on experience on algorithms that exploit parallel processing in a distributed mode. Such algorithms are in common use. (Here I need some numbers from reliable sources.)

We would like to extend our research in the learning of students of computer science about mathematical proofs \cite{smith2013mathematization,smith2014computer,smith2016categories} to mathematical proofs about software executing in distributed systems, as are given in Lynch's Distributed Algorithms \cite{lynch1996distributed}.
\subsection{Visibility}
By providing a facility for distributed processing in our department, and making use of it available to any member of the Connecticut State Universities, we increase the visibility of CCSU's CS department. Conceivably in the future we might extend the availability of this facility more broadly.
\subsection{Research stature}
We hope to use this capability to support our research in computational medicine. We have reached out to the Biomedical Science department, in connection with a planned graduate specialization in computational support for biomedicine. We hope that this facility will support cooperation between our faculty members. Moreover, we hope to support existing projects at the larger scale available with distributed processing, including the search for biomarkers in the relatively restricted contexts of blood tests, urine tests and breath tests. We hope to develop opportunities for commercial activity in the development of these tests, which correspond to manufacturable test kits. Because of the anticipated support for commercial activity, we hope to develop external funding.

\section{instructions for proposal preparation and submission / Proposal Narrative}%1200 words in 5 pages TimesNewRoman 12 point, double spaced paragraphs, one inch margins all four edges.
%appendix with graphics, only when crucial
\subsection{Cover}%page does not count in 5 pages
%see appendix a.1, there is something about expertise
%state no people or other verterbrates
\noindent\fbox{%
	\parbox{\textwidth}{
		Faculty Rank of Principal Contact: \\
Last Name: \\
First Name: \\
University: \\
Department: \\
Funding Request: \$\\
Is this a Joint Proposal? \Checkedbox Yes    $\square$ No\\
If Yes, please fill in inormation for co-proposers (add separate sheets as necessary ):\\
Name         Rank       Department

Is this a Continuation Project?   Yes   x No       If Yes, you must complete Appendix A.2\\
E-mail of Principal Contact:                      Phone Number of Principal Contact:\\
Campus Address of Principal Contact:
}}
Please \underline{select one} disciplinary group category in which this project best fits:\\
 $\square$    Fine Arts and Humanities           $\square$   Social Sciences, Buisiness and Education\\
\Checkedbox   Life and Physical Sciences, Mathematics, Computer Science, Engineering and Technology

\noindent Please \underline{select one} research focus area in which this project best fits:\\
 $\square$ Creation of new knowledge     \Checkedbox Application of disciplinary/multidisciplinary knowledge, methodologies and/or insights     $\square$ Production of creative works  $\square$ Research in student learning

Project Title:

	%email to grants@ccsu.edu as pdf by Feb. 1, 5pm
	%named DISCIPLINARY GROUPING_LAST NAME_INITIAL_CAMPUS.FILEEXTENSION
	
	%For example – FAH_SMITH_J_CCSU.pdf
	%LPMCET_Smith_T_CCSU.pdf
	
	
	


\begin{center}
	\textbf{ABSTRACT (Limit: 100 words)}
	\begin{abstract}
		We wish to install a distributed programming environment based upon Hadoop, which will serve several purposes. First, we can teach students with hands-on experience of distributed computing. Second, we can support large datasets, and parallelization of suitable algorithms; this includes algorithms for computational medicine. We can provide more extensive support for our statistical calculations (which support biomarker discovery), and carry them out faster, in the distributed programming framework, provided we have multiple machines. Biomarkers are helpful in medical diagnosis. Kits that test for biomarkers are a possible product suitable for manufacture. %91 words
	\end{abstract}
\end{center}





\noindent \textbf{IRB/IACUC Statement}\\
(If ``yes'' to either question please see Section 5, p. 3 of the program guidelines)\\
\begin{tabular}{ccl}
	YES  &    NO & \\
	\Checkedbox & $\square$  &Does your research involve human beings as research subjects?\\
	$\square$  &\Checkedbox &Does your research involve vertebrate animals?\\
\end{tabular}



\underline{\textbf{Sign-Off Statement}} (To be signed individually be each faculty applicant. Please add separate sheets as needed)\\
\textit{I hereby acknowledge my understand that the lack of compliance with the porposal format and other requirements spelled out in the CSU - AAUP Faculty Research Grant Guidelines for the Spring 2018 Competition may result in the proposal being disqualified without review.}\\

\begin{tabular}{cc}
 
$\rule{7cm}{0.15mm}$	& $\rule{4cm}{0.15mm}$ \\ 
 
Signature of Permanent, Full-Time Faculty	&  Date\\ 
 
$\rule{7cm}{0.15mm}$	&  $\rule{4cm}{0.15mm}$\\ 
 
Signature of Permanent, Full-Time Faculty	&  Date\\ 
	 
$\rule{7cm}{0.15mm}$	& $\rule{4cm}{0.15mm}$\\ 
	 
Signature of Permanent, Full-Time Faculty	&  Date\\ 
	 
\end{tabular} 

       
\newpage

%\begin{center}
%	\textbf{Appendix A.2:}\\
% \textbf{REPORT ON CONTINUATION PROJECT}\\
%	 \textbf{CSU-AAUP FUNDED RESEARCH}
%\end{center}
%Proposals requesting support for continuation of work previously funded by this program must include a summary herein.  This summary should include compelling evidence of the impact of the research conducted, such as publications in peer reviewed journals, securing of external funding for the expansion/continuation of the work, presentations at professional conferences, performances or exhibits, book publications, etc.  The Report on Continuation Project should be no more than two pages.


%\noindent\textit{Note:  Lack of compliance with programmatic or fiscal reporting requirements related to this program will be handled in accordance with university procedures.}

%\newpage

\begin{center}
 \textbf{Appendix A.3:}\\
 	\textbf{BUDGET AND BUDGET JUSTIFICATION FORM}\\
\end{center}

\begin{table}[H]
	\caption{\textbf{2018 - 2019 CSU - AAUP Faculty Research Grant}}
	%.  In the “Brief Justification” column please provide a general for each cost, (e.g., name equipment purchased, provide approximate number of hours and hourly rate for student assistants). In the space below, please provide up to about 100 words of text with further details making a case for the proposed expenditures
\begin{tabular}{|p{4cm}|p{3cm}|c|}
	\hline 
Budget Item	&Amount &  Brief Justification\\ 
 	&(No Cents) &   \\ 
	\hline 
Faculty Stipend	&  &  \\ 
	\hline 
Support Services *	&  &  \\ 
	\hline 
Supplies and Equipment	&  &  \\ 
	\hline 
Travel	&  &  \\ 
	\hline \hline
 \multicolumn{1}{|r|}{\textbf{Total}}	&  & \textbf{N/A} \\ 
	\hline 
\end{tabular} \\
* For definition see Section 9.4 of the ``Collective Bargaining Agreement between Connecticut State University, American Association of University Professors and Board of Regents for Connecticut State Colleges \& Universities System, August 26, 2016 – August 26, 2021'', Section 9.4, pp. 56-57.
\end{table}


\subsection{Significance}%audience is informed generalist in the discipline grouping (computer science, life and physical sciences)
We have a three-fold purpose: 
\begin{itemize}
	\item Provide distributed processing infrastructure for our own research in computational medicine
	\item Provide distributed processing infrastructure for teaching students about distributed algorithms
	\item Conduct research on biomarkers that is intended to lead to manufacturable test kits
\end{itemize}

\subsubsection{Outline of Related Research}%selves and others

\subsection{Work Plan}%show soundness
The software infrastructure, a modification of Hadoop, has been developed by Roland DiPratti.
The plan is to identify machines onto which we can install this software, to make an operating facility.
We plan to use machines obtained separately from the grant.
Then we plan to install and verify the installation of the modified Hadoop.
Then we plan to install software used by the application, including Python and R, which are generally useful packages.
Then we plan to install software used by the application, a survival analysis package of R, which is more focussed in purpose, though of interest to both insurance and medical applications.
Then we plan to load and execute our first application, which is expected to be the biomarker software, which uses Python and R. 
 We plan to apply the biomarker software to larger datasets than those to which it has been applied so far; this will be facilitated by the modified Hadoop software.
 \subsection{Outcomes and Reporting}
 
 It is certainly our intention to submit the results of our biomarker research to a journal such as BMC Bioinformatics (\url{https://link.springer.com/journal/12859}). This research is related to our previous research \cite{ammar2013developing}.
 
 We intend to submit the results of our research in approaches to teaching distributed programming to ICER and Koli Calling, which have accepted our work previously.
 
% \subsection{Report on Previous AAUP, if any}%2 pages, does not count in 5 page budget
 \subsection{Budget proposal}
 %realistic, appendix a3, clearly stated, justified, consistent with purpose, work plan, outcomes
 %pages do not count in page count
 
 \subsection{CVs}
 %do not count in page count, cannot be more than 2 pages per person
 \subsection{Joint proposal individual contributions and level of participation}
 %adds one page / 240 words per additional faculty member
 \subsection{Bibliography} %does not count in page count
 \subsection{Optional Appendices} %do not count in page count
 
 

\section{proposal review criteria}
\begin{itemize}
	\item coversheet abstract
	\item signoff statement %see appendix a.1
	\item proposal narrative
\end{itemize}

%\section{calendar for proposal submission}
%\section{review process and announcement of awards (consistent with contractually established schedules)}
%\section{cover sheet to be attached to each proposal}

\bibliographystyle{alpha}
\bibliography{lit}
\end{document}
